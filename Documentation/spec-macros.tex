% The "type" part of @defthing.
\newcommand{\FloatRight}[1]{\nobreak\hfill\enspace #1}
\newcommand{\Param}[1]{\mbox{#1}}
% For fancy looking definition section headers.
\newcommand{\SmallCaps}[1]{\textsc{#1}}

\renewenvironment{SInsetFlow}{%
  \hfill\begin{minipage}{\dimexpr\textwidth-2em}%
}{
  \end{minipage}\\
}

\newenvironment{Definitions}{%
  \vspace{0.5em}
  \begin{minipage}{\textwidth}
    \setlength{\parskip}{0.5em}
    \noindent{\rule{\linewidth}{0.4pt}}
}{
  \end{minipage}
}

% Definition lines
% Thanks to https://tex.stackexchange.com/questions/312581/how-to-calculate-width-of-remaining-part-of-line
% for more or less showing how to get this layout working.

\newsavebox{\defunnamebox}
\newlength{\defunnamewidth}

\newcommand{\DefunName}[1]{%
  \sbox{\defunnamebox}{\strut \textbf{#1}}%
  \global\defunnamewidth=\wd\defunnamebox
  \usebox{\defunnamebox}%
}
\newcommand{\LambdaList}[1]{%
  \parbox[t]{\dimexpr\columnwidth-\defunnamewidth-\fboxsep}{
    \raggedright\strut\emph{#1}
  }%
}

\newenvironment{Landscape}{%
  \begin{landscape}
}{
  \end{landscape}    
}

\newenvironment{CenteredContainer}{}{}

% Okay, not really centering, but I don't know if I really want centering,
% or just a big indent.
\newenvironment{CenteredBlock}{%
  \hfill\begin{minipage}{\dimexpr\textwidth-6em}\raggedright
}{
  \end{minipage}
}
  
\setlength{\parindent}{0pt}

% Have each section start on a new page.
\let\oldsection\section
\renewcommand\section{\clearpage\oldsection}
\setlength{\parskip}{0.5em}